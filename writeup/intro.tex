\documentclass[12pt, oneside, a4paper]{article}

% for math symbols
\usepackage{amsmath}
\usepackage{amssymb}
% for inserting images
\usepackage{graphicx}
% for algorithm pseudocode
\usepackage{algorithm}
\usepackage{algpseudocode}
% for tables
\usepackage{tabularx}
% for implementation of the array and tabular environments
\usepackage{array}
% Control float placement. Defines a \FloatBarrier command
\usepackage{placeins}
% for derivative commands
\usepackage{physics}
% for multi level lists
\usepackage{outlines} 
% for links in text
\usepackage[colorlinks=true,linkcolor=blue,urlcolor=black,bookmarksopen=true]{hyperref}
% for contents after pdf is formed
\usepackage{bookmark}
% for caption
\usepackage{caption}
% for code
\usepackage{listings}
% for pictures
\usepackage{tikz}


\def\nudge{.5}
\tikzset{axis/.style={ultra thick, black, -latex, shorten <=-\nudge cm, shorten >=-2*\nudge cm}}
\tikzset{line/.style={thick,green}}

\hypersetup{%
  colorlinks=true,
  linkcolor=blue,
  linkbordercolor={0 0 1}
}

\newcommand{\vect}[1]{\ensuremath{\mathbf{#1}}}
\newcommand{\vt}[1]{\ensuremath{\mathbf{#1}}}
\newcommand{\uline}[1]{\underline{#1}}
\newcommand{\tb}[1]{\textbf{#1}}

\DeclareUnicodeCharacter{2212}{\textendash}

\graphicspath{ {./images/} }

\title{CSDS 341: Final Project Initial Report\\ Questionnaire Website}
\author{Oleksii Fedorenko, David Frost, Matthew Garcia, Preeti Naik}

\begin{document}
    \maketitle
    \section{Background}
    Questionnaires and surveys are very important for understanding and shaping the world. The United States uses the Census and the
    American Community Survey to understand how the country is changing and to make decisions about how to allocate its
    resources. Researchers use questionnaires for scientific studies to understand both the social and natural worlds.
    Psychologists in particular make great use of Likert scales in questionnaires to help build our collective knowledge.
    Until recently, however, the technologies for creating, collecting, and analyzing questionnaires did not exist.
    The Internet now allows for the possibility of giving people the power to create their own questionnaires and distribute
    them easily. Our project allows for the convenient creation of questionnaires and answering of questionnaires. Additionally,
    we allow for analysis of data to an advanced degree that is not available in most extant questionnaire and survey websites.
    \section{Data Description and Schemas}
    \textbf{Entities}
    \\
    Users(user\_id       INT,\\
          first\_name    CHAR(30),\\
          last\_name     CHAR(30),\\
          email         CHAR(50)\\
          PRIMARY KEY(user\_id)\\
    )
    \\
    \\
    This entity represents users of the questionnaire website.
    All types of users, including people who create questionnaires
    or answer questionnaires, are in this table.
    \section{ER Diagram}
\end{document}